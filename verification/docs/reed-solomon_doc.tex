\documentclass[a4paper]{article}

\usepackage[a4paper, left=2.5cm, right=2cm]{geometry} % A4 paper size and thin margins

\usepackage[table,xcdraw]{xcolor} % Required for specifying custom colours
\definecolor{grey}{rgb}{0.9,0.9,0.9} % Colour of the box surrounding the title
\definecolor{LightBlue}{rgb}{0.69,0.768,0.95} % Colour of the box surrounding the title

\usepackage[utf8]{inputenc} % Required for inputting international characters
\usepackage[T1]{fontenc} % Output font encoding for international characters
%\usepackage[sfdefault]{ClearSans} % Use the Clear Sans font (sans serif)
\usepackage{XCharter} % Use the XCharter font (serif)

\usepackage{graphicx}
\usepackage{draftwatermark} % adds draft watermark
\SetWatermarkLightness{0.85}
\SetWatermarkText{Confidencial}
\SetWatermarkScale{0.7}

\usepackage{vhistory}

\usepackage{indentfirst}
\usepackage{longtable}


\usepackage{fancyhdr}
\pagestyle{fancy}
\rhead{\leftmark}
\lhead{Guia de Verificação - Bloco reed-solomon}

\usepackage[portuguese]{babel}

\usepackage{adjustbox}
\usepackage{multirow}
\usepackage{listings}


\begin{document}

%----------------------------------------------------------------------------------------
%   TITLE PAGE
%----------------------------------------------------------------------------------------

\begin{titlepage} % Suppresses displaying the page number on the title page and the subsequent page counts as page 1
    
    
    
    %------------------------------------------------
    %   Grey title box
    %------------------------------------------------
    
    \begin{figure}
    \centering
    \begin{minipage}{0.45\textwidth}
        \centering
        \includegraphics[width=0.9\textwidth]{logo.jpg} % first figure itself
    \end{minipage}\hfill
    \begin{minipage}{0.45\textwidth}
        \centering
        \includegraphics[width=0.9\textwidth]{logo_virtus.png} % second figure itself
    \end{minipage}
\end{figure}
    
    \vspace{8cm}
        \parbox[t]{0.93\textwidth}{ % Outer full width box
            \parbox[t]{0.91\textwidth}{ % Inner box for inner right text margin
                \raggedleft % Right align the text
                \fontsize{36pt}{40pt}\selectfont % Title font size, the first argument is the font size and the second is the line spacing, adjust depending on title length
                \vspace{0.7cm} % Space between the start of the title and the rs_top of the grey box
                
                \textbf{Guia de Verificação \\
                Bloco reed-solomon}
                
                \vspace{0.7cm} % Space between the end of the title and the bottom of the grey box
            }
        }
    
    \vfill % Space between the title box and author information
    
    %------------------------------------------------
    %   Author name and information
    %------------------------------------------------
    
    \parbox[t]{0.93\textwidth}{ % Box to inset this section slightly
    
        \raggedleft % Right align the text
        \hfill\rule{0.2\linewidth}{1pt} \\[4pt]% Horizontal line, first argument width, second thickness
        \large % Increase the font size
        {\Large Laboratório de Excelência em Microeletrônica}\\[4pt] % Extra space after name
        Virtus\\
        Universidade Federal de Campina Grande\\[4pt] % Extra space before URL

        \hfill\rule{0.2\linewidth}{1pt}% Horizontal line, first argument width, second thickness
    }
    
\end{titlepage}

\title{Guia de Verificação - Bloco reed-solomon}
%---------------------------------------------------------------------------------------


    
\pagebreak
\hspace{0pt}
\vfill
\begin{center}
\hfill\rule{\linewidth}{1pt} \\[4pt]
    Este é um documento \textbf{CONFIDENCIAL} \\
    é vetado qualquer reprodução ou divulgação não autorizada.\\[4pt]
    \hfill\rule{\linewidth}{1pt} \\[4pt]
\end{center}
\vfill
\hspace{0pt}
\pagebreak

%---------------------------------------------------------------------------------------

% Start of the revision history table
\begin{versionhistory}
  \vhEntry{1.0}{10.02.19}{José Iuri}{created}
\end{versionhistory}

\newpage

\tableofcontents

\newpage

\section{Introdução}
\subsection{Objetivos}

\par
Esse relatório tem o objetivo de apresentar o desenvolvimento de uma arquitetura de \textit{testbench} para a verificação funcional baseada na metodologia UVM do bloco reed-solomon.

\subsection{Temos Utilizados}

\begin{table}[h]
\centering
\begin{tabular}{|c|c|}
\hline
\rowcolor[HTML]{27378F} 
{\color[HTML]{FFFFFF} Termo} & {\color[HTML]{FFFFFF} Significado} \\ \hline
DUT                          & Device Under Test                  \\ \hline
UVM                          & Universal Verification Methodology \\ \hline
RM                           & Reference Model                    \\ \hline
RTL                          & Register Transfer Model            \\ \hline
\end{tabular}
\end{table}


\subsection{Nomenclaturas e Representação de Bits e Amostras}

\begin{table}[h]
\centering
\resizebox{\textwidth}{!}{%
\begin{tabular}{|
>{\columncolor[HTML]{27378F}}c |l|l|l|}
\hline
\cellcolor[HTML]{27378F}{\color[HTML]{FFFFFF} }                                                                                       & MSS   & Most Significant Sample                                                                             &                                                                                                                                     \\ \cline{2-3}
\multirow{-2}{*}{\cellcolor[HTML]{27378F}{\color[HTML]{FFFFFF} Sample Representation}}                                                & LSS   & Least Significant Sample                                                                            & \multirow{-2}{*}{(MSS:LSS)}                                                                                                         \\ \hline
\cellcolor[HTML]{27378F}{\color[HTML]{FFFFFF} }                                                                                       & MSB   & Most Significant Bit                                                                                &                                                                                                                                     \\ \cline{2-3}
\multirow{-2}{*}{\cellcolor[HTML]{27378F}{\color[HTML]{FFFFFF} Bit Representation}}                                                   & LSB   & Least Significant Bit                                                                               & \multirow{-2}{*}{{[}MSB:LSB{]}}                                                                                                     \\ \hline
\cellcolor[HTML]{27378F}{\color[HTML]{FFFFFF} }                                                                                       & NBW   & Total Number of Bits - Word length                                                                  &                                                                                                                                     \\ \cline{2-3}
\cellcolor[HTML]{27378F}{\color[HTML]{FFFFFF} }                                                                                       & NBI   & Number of Integer portion length                                                                    &                                                                                                                                     \\ \cline{2-3}
\cellcolor[HTML]{27378F}{\color[HTML]{FFFFFF} }                                                                                       & QUANT & Quantization by default: TRN                                                                        &                                                                                                                                     \\ \cline{2-3}
\cellcolor[HTML]{27378F}{\color[HTML]{FFFFFF} }                                                                                       & OVFLW & Overflow by default: WRAP                                                                           &                                                                                                                                     \\ \cline{2-3}
\multirow{-5}{*}{\cellcolor[HTML]{27378F}{\color[HTML]{FFFFFF} \begin{tabular}[c]{@{}c@{}}Fixed Point\\ Representation\end{tabular}}} & NBITS & Number of saturated bits, only used for overflow mode and specifies how many bits will be saturated & \multirow{-5}{*}{\begin{tabular}[c]{@{}l@{}}\textless WL,IWL{[},QUANT{]}\\ {[},OVFLW{]}{[},NBITS{]} \textgreater bits\end{tabular}} \\ \hline
\cellcolor[HTML]{27378F}{\color[HTML]{FFFFFF} }                                                                                       & '1'   & \multicolumn{2}{l|}{Bit asserted}                                                                                                                                                                                                         \\ \cline{2-4} 
\cellcolor[HTML]{27378F}{\color[HTML]{FFFFFF} }                                                                                       & '0'   & \multicolumn{2}{l|}{Bit de-asserted}                                                                                                                                                                                                      \\ \cline{2-4} 
\cellcolor[HTML]{27378F}{\color[HTML]{FFFFFF} }                                                                                       & 'x'   & \multicolumn{2}{l|}{Bit don't care}                                                                                                                                                                                                       \\ \cline{2-4} 
\cellcolor[HTML]{27378F}{\color[HTML]{FFFFFF} }                                                                                       & R     & \multicolumn{2}{l|}{Readable bit}                                                                                                                                                                                                         \\ \cline{2-4} 
\cellcolor[HTML]{27378F}{\color[HTML]{FFFFFF} }                                                                                       & W     & \multicolumn{2}{l|}{Writable bit}                                                                                                                                                                                                         \\ \cline{2-4} 
\cellcolor[HTML]{27378F}{\color[HTML]{FFFFFF} }                                                                                       & U     & \multicolumn{2}{l|}{Unimplemented bit, read as 0/1/x}                                                                                                                                                                                     \\ \cline{2-4} 
\multirow{-7}{*}{\cellcolor[HTML]{27378F}{\color[HTML]{FFFFFF} Logic Values}}                                                         & -n    & \multicolumn{2}{l|}{Value at POR}                                                                                                                                                                                                         \\ \hline
\end{tabular}%
}
\end{table}

\subsection{Referências}

\pagebreak


\section{DUT}

\subsection{Descrição}

Xxxxx xxxxxxx.


\subsection{Funcionalidades}

\begin{itemize}
    \item xxxxxxxxx;
    \item yyyyyy;
    \begin{itemize}
        \item zzzzzzzzzzz;
    \end{itemize}
\end{itemize}

\subsection{Parâmetros}

\begin{longtable}{|l|l|l|l|}
\hline
\rowcolor[HTML]{27378F} 
{\color[HTML]{FFFFFF} Parâmetro} & {\color[HTML]{FFFFFF} Tipo} & {\color[HTML]{FFFFFF} Valor padrão} & {\color[HTML]{FFFFFF} Descrição}                                                                                        \\ \hline
NBW\_OUT                         & integer                     & 8                                   & Número de bits de saída.                                                                                       \\ \hline
\end{longtable}

\subsection{Interface}

% Please add the following required packages to your document preamble:
% \usepackage{graphicx}
% \usepackage[table,xcdraw]{xcolor}
% If you use beamer only pass "xcolor=table" option, i.e. \documentclass[xcolor=table]{beamer}
\begin{table}[h]
\centering
\resizebox{\textwidth}{!}{%
\begin{tabular}{|l|l|l|l|l|}
\hline
\rowcolor[HTML]{27378F} 
\multicolumn{1}{|c|}{\cellcolor[HTML]{27378F}{\color[HTML]{FFFFFF} Sinal}} & \multicolumn{1}{c|}{\cellcolor[HTML]{27378F}{\color[HTML]{FFFFFF} Tamanho}} & \multicolumn{1}{c|}{\cellcolor[HTML]{27378F}{\color[HTML]{FFFFFF} I/O}} & \multicolumn{1}{c|}{\cellcolor[HTML]{27378F}{\color[HTML]{FFFFFF} Sync}} & \multicolumn{1}{c|}{\cellcolor[HTML]{27378F}{\color[HTML]{FFFFFF} Descrição}}                                                                                                                                           \\ \hline
rst\_async\_n                                                              & 1                                                                           & I                                                                       & Async                                                                    & Reset, assíncrono e BAIXO ativo                                                                                                                                                                                         \\ \hline
clk                                                                        & 1                                                                           & I                                                                       & clk                                                                      & Clock funcional                                                                                                                                                                                                         \\ \hline
\end{tabular}%
}
\end{table}

\pagebreak

\section{Arquitetura Proposta}

Um conjunto de testes foi desenvolvido no intuito de verificar as funcionalidades do bloco de forma isolada. Os testes são aplicados por meio de um \textit{testbench} escrito em \textit{Systemverilog} baseado na metodologia UVM ilustrado na Figura \ref{tb}.

\begin{figure}[h]
    \centering
        \centering
        \includegraphics[width=\textwidth]{diagrama.png} % first figure itself
        \caption{Arquitetura do testbench do reed-solomon}
        \label{tb}
\end{figure}

\subsection{Arquitetura do \textit{testbench}}

\begin{itemize}
    \item Top
    
        O nível superior do testbench cria instâncias do teste a ser executado, o DUT, o RM e a interface contendo os sinais que conduzirão valores de/para o DUT e o RM e os demais compontentes do ambiente UVM criado.
        
    \item DUT
    
        
        
        \newpage
        
    \item RM
    
        
        
    \item Comparador
    
        
        
    \item Inteface
    
        
    
    \item agent\_in
        
        
        
    \item agent\_out
        
        
    
    \item Test
        
        
    
\end{itemize}

\subsection{Execução do \textit{Testbench}}

O ambiente de verificação pode ser obtido em:

\vspace{5pt}

\lstset{language=bash}
\begin{lstlisting}[backgroundcolor=\color{grey}][basicstyle=\small]
commit 3213213213546510321358465 (origin/reed-solomon/Verification/tb)
Author: Fulano Silva <fulano.silva@embedded.ufcg.edu.br>
Date:   Tue Feb 12 11:11:04 2019 -0300

    Ultimo commit funcional.

\end{lstlisting}

\vspace{5pt}

Para rodar a simulação executar o Makefile. As opções de execução podem ser checadas no README.

\vspace{5pt}


\lstset{language=bash}
\begin{lstlisting}[backgroundcolor=\color{grey}][basicstyle=\small]
options:

aqui fica o help do arquivo de README.

\end{lstlisting}

\vspace{5pt}

\pagebreak
\section{Testes}

\subsection{Simple Test}

Teste simples com dados aleatórios de entrada de acordo com características previamente configuradas e entradas de configuração fixa.
 
\pagebreak

\section{Resultados}

\subsection{Simple Test}

Resultados.

\end{document}
